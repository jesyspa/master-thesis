\chapter{Introduction}

% Outline:
% Example of an encryption scheme.
% Problem: how do we quantify how secure this is?
% Example attack: Eavesdropping.
% Example attack: Chosen Plaintext Attack
% Formalisation in Agda
% Existing work



For example, consider the One-Time Pad encryption scheme for bitstrings of length $n$: to generate a key we take a
random bitstring $k$ of length $n$, and to encrypt or decrypt the message $m$ we take the XOR $k \oplus m$.  One
`game' we can play goes as follows: we generate a key $k$ and let the adversary choose two messages $m_1$ and $m_2$.
We then flip a coin and encrypt $m_1$ if it is heads and $m_2$ if it is tails, and give the encrypted message to the
adversary, who must tell us whether the coin landed heads or tails.  If the adversary answers correctly, they win;
otherwise, they lose.

We say that the One-Time Pad encryption scheme is secure to Eavesdropping if any adversary has at most 1/2
probability of winning in the above game.  It turns out that this does indeed hold for the One-Time Pad.  A stronger
version of this condition we may want to look at is whether this is still the case if the adversary may first
encrypt any number of strings; this is called security against a Chosen-Plaintext Attack, and it turns out that the
One-Time Pad is \emph{not} secure against it.

To prove such a property, one typically constructs a sequence of games, where the probability of an adversary
winning only improves by at most some very small value in each transition.  The final game in the construction is
typically one where the probability is clear: for example, if whether the adversary wins or loses does not depend on
the choice made by the adversary.

The aim of this research project is to create a system in which both encryption schemes and such proofs of security
can be expressed and automatically verified.  This system will be built on top of the Agda programming language,
allowing the prover to use other Agda libraries, and allowing this system to be used in larger Agda programs.

\section{Formalised Proofs}

\section{Existing Work}

The game-playing approach to cryptographic proofs is an existing and popular technique~\cite{codebasedgames}.

There are several systems for expressing cryptographic proofs in Coq or in special-purpose
languages\footnote{EasyCrypt, FCF}.  However, these all put the imperative style that cryptographic algorithms are
typically written in at the forefront, which shapes the remainder of the system.  We intend to put more focus on the
search for an elegant type-theoretic formulation of these properties, which should make the resulting proofs more
elegant.

TODO: Stochastic lambda calculus papers, dist monad papers?
