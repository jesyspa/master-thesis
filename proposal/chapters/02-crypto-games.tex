\chapter{Game-Playing Proofs}

Now that we have sketched the overarching ideas of the thesis, let us return from the formalised treatment and see what
techniques and results come up when studying cryptographic algorithms via games.  The goal of this chapter is to
familiarise the reader with a number of games that are important in cryptography, together with some approaches that can
be used to prove results about them.

% Outline:
% Historic background
% Notion of an adversary [later: of an oracle]
% Standard game definitions

\section{Formal Definition}

A game is always parametrised by two things: a scheme and an adversary.  It defines, using a sequence of interactions, a
relation 

A game is a procedure describing a sequence of interactions with an adversary, concluding in an event that indicates
whether the adversary wins or loses the game.

An adversary is an algorithm that the game is parametrised over.



\section{Important Games}

