\chapter{Probability in Agda}

In the previous chapter, we specified a language for expressing games and postulated certain rewrite rules that may be
used to show correctness.  We will now show that this language has a meaningful computational interpretation.  We will
start by postulating a certain numeric type that is sufficient to represent probabilities, and then specify what
properties a probability distribution type should satisfy.

At the moment, we have only successfully implemented the stateless interpretation of the syntax in the previous chapter.
However, a state monad transformer applied to the construction in this chapter should suffice to add the |getAdvState|
and |putAdvState| operations.

\section{Probability}

We will need a type for representing probability values.  There are two important questions to address:
\begin{enumerate}
    \item Should probabilities lie strictly in the $[0, 1]$ interval?
    \item Should the real numbers (or the real unit interval) be permitted as a model?
\end{enumerate}

We have chosen to answer `no' to both of these questions.  It is very convenient for probabilities to be a group under
addition, which the interval $[0, 1]$ is not.  While an implementation that distinguishes the types of probabilities,
differences between probabilities, and sums of probabilities would be an interesting project in itself, it would add
little to the logic we are studying.


\section{Distributions: Abstract Specification}

\section{List-Based Distributions}

\section{Continuation-Based Distributions}


% Outline:
% Probability in mathematics.
% Postulate a Probability type.
% Two common ways of expressing a probability: list of probabilities or measure-theoretic.
% Valuation of syntax.
