\chapter{Probability in Agda}

In the previous chapter, we specified a language for expressing games and postulated certain rewrite rules that may be
used to show correctness.  We will now show that this language has a meaningful computational interpretation.  We will
start by postulating a certain numeric type that is sufficient to represent probabilities, and then specify what
properties a probability distribution type should satisfy.

At the moment, we have only successfully implemented the stateless interpretation of the syntax in the previous chapter.
However, a state monad transformer applied to the construction in this chapter should suffice to add the |getAdvState|
and |putAdvState| operations.

\section{Probability}

In order to discuss how probability distributions can be sampled, we will need a type for representing probability
values.  A key question we must ask ourselves is whether this type should \emph{only} be able to represent values in the
$[0, 1]$ interval, or whether values that are not probabilities should also be representable.

We have chosen to go for the later options.  It is very convenient for probabilities to be a group under addition, which
the interval $[0, 1]$ is not.  While an implementation that distinguishes the types of probabilities, differences
between probabilities, and sums of probabilities would be an interesting project in itself, it would add little to the
logic we are studying.

Instead, we require that the type of probabilities |Q| be an ordered ring with negative powers of two, in the sense that
there is a function |negpow2 : Nat -> Q| such |negpow2 0 == 1| and |2 * negpow2 (n+1) = negpow2 n|.\footnote{Where by 0
and 1 we mean the corresponding values in the ring structure, and |2 = 1 + 1|.}  This suffices to implement
probabilities that occur when the only source of randomness are uniform distributions over sets with size a power of
two.  If we wish to add a |Repeat| combinator, as discussed in the previous chapter, then requiring that |Q| be an
ordered field will be necessary.

We have so far not implemented this type, but given that the rationals satisfy all these assumptions in a constructive
manner, we believe that it should have an implementation.

Note that since an ordered ring necessarily has a \emph{total} order, the real numbers are \emph{not} a model of
probability, since real equality is not constructively decidable.  We have yet to see whether we will ever make use of
this decidability.  In any case, it is unclear that using the real numbers as |Q| would have any advantages, since all
probabilities we arrive at are by construction rational.

\section{Distributions: Abstract Specification}

\section{List-Based Distributions}

\section{Continuation-Based Distributions}


% Outline:
% Probability in mathematics.
% Postulate a Probability type.
% Two common ways of expressing a probability: list of probabilities or measure-theoretic.
% Valuation of syntax.
