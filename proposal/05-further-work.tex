\chapter{Research Plan}

% Outline:
% Oracles, more games
% Security up to some epsilon
% Security relative to a hard problem
% Polynomial-time oracles

We have now created a basic version of the system in which we can reason about the security of the One-Time Pad
construction described above, and show that it is secure with respect to the Eavesdropper attack.  The proofs are
incredibly verbose, but the properties we are looking for hold.  

Further work is necessary to express more complicated games, for example where the adversary has access to a
(stateful) oracle\footnote{TODO: define oracle}.  We also expect to be able to greatly improve the readability of
proofs and reduce the need for repetition in them.

Momentarily, the the formulation of the security conditions require security to be absolute---it must be impossible
for the adversary to do better than exactly random chance.  It is, in fact, of more interest to allow for some small
`advantage', as long as it is asymptotically vanishing or at least bounded by some small constant.  Finding elegant
ways to express this will be a considerable part of the project.

Finally, in practice it is often interesting to consider only polynomial-time adversaries.  Due to the opaque nature
of functions in Agda this will be a challenge to implement, but would be an interesting endeavour if there is
time remaining.

\section{Timetable and Planning}

TODO: Write neat text.
\begin{itemize}
    \itemsep0em
    \item February-March: expressing games, expressing encryption schemes, expressing proof techniques
    \item April-May: proving stronger results, start writing thesis (prior knowledge)
    \item June: write thesis (my results), prove leftover postulates, express things like polynomial-time adversaries
    \item July: integrate last results into thesis, correct last
\end{itemize}

