\documentclass{article}

\usepackage{amsmath}
\usepackage{amssymb}
\usepackage{amsfonts}
\usepackage{amsthm}
\usepackage{tikz-cd}
\usepackage[margin=2.5cm]{geometry}

\newtheorem{theorem}{Theorem}
\newtheorem{question}{Question}
\newtheorem{lemma}[theorem]{Lemma}
\newtheorem{corollary}[theorem]{Corollary}
\theoremstyle{definition}
\newtheorem{definition}[theorem]{Definition}

\newcommand{\mc}[1]{\mathcal{#1}}
\newcommand{\iss}[1]{\mathcal{#1}_S}
\newcommand{\isc}[1]{\mathcal{#1}_C}
\newcommand{\isr}[1]{\mathcal{#1}_R}
\newcommand{\ist}[1]{\mathcal{#1}_t}

\DeclareMathOperator{\Fam}{Fam}

\title{\vspace{-2cm}Memory Manipulation}

\begin{document}
    \maketitle

    Scratchpad document for presenting the memory example.

    The basic example is an interaction structure that tracks the amount of memory currently available and has
    operations for allocating, deallocating, writing, and reading memory.  We can show how this can be implemented in a
    straightforward way in an indexed state monad.  It may also be interesting to show how we can implement this in a
    non-indexed state monad.

    A minor variation on this example is to allow the use of arbitrary indices in the read and write operations, losing
    some type safety.
\end{document}
