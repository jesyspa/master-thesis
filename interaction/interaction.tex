\documentclass{article}

\usepackage{amsmath}
\usepackage{amssymb}
\usepackage{amsfonts}
\usepackage{amsthm}
\usepackage{tikz-cd}
\usepackage[margin=2.5cm]{geometry}

\newtheorem{theorem}{Theorem}
\newtheorem{question}{Question}
\newtheorem{lemma}[theorem]{Lemma}
\newtheorem{corollary}[theorem]{Corollary}
\theoremstyle{definition}
\newtheorem{definition}[theorem]{Definition}

\DeclareMathOperator{\Fam}{Fam}

\title{\vspace{-2cm}Interaction Structures}

\begin{document}
    \maketitle

    In this document, we will present interaction structures and show their role in our development.  We use, amongst
    other things, the work of Hancock (Ordinals and Interactive Programs).  Our development is phrased in type
    theoretical term terms, with care taken to avoid the use of the law of excluded middle, and so we expect this to be
    formalisable in Agda in a straightforward manner.

    When talking of a family of types $B$ indexed over $A$, we will use $B$ to refer both to the family and to the
    coproduct $\sum_{a \in A} B_a$.  Essentially, this corresponds to the view that an indexed family is a type $B$
    together with a projection $B \to A$.

    \section{Interaction Structures}

    \begin{definition}
        An \emph{interaction structure} is a tuple $(S, C, R, t)$, where $S$ is a type, $C$ is an $S$-indexed family of
        types, $R$ is a $C$-indexed family of types, and $t$ is a transition function $R \to S$.
    \end{definition}

    Note that $R$ has an implicit dependency on $S$ (corresponding to the composition of the projections).

    According to Hancock, an equivalent presentation exists as follows: let $\Fam(X)$ be the type of functions into $X$,
    indexed by their domain. An interaction structure $(S, C, R, t)$ corresponds to a function $S \to \Fam^2(S)$.
    Writing this out, the type of interaction structures corresponds to the type
    \[
        \sum_S (S \to (\sum_C (C \to \sum_R (R \to S)))).
    \]

    The transformation is given as follows:
    \[
        (S, \lambda s.\, (C_s, \lambda c.\, (R_c, \lambda r.\, t(r)))).
    \]
    We thus see that an interaction structure is a coalgebra on the functor $\Fam^2$.

    \subsection{Coalgebra Morphisms}

    Since coalgebras have a morphism
    structure, we can consider interaction structures to form a category.  Explicitly, a morphism from a coalgebra $\phi
    : S \to \Fam^2(S)$ to a coalgebra $\psi : S' \to \Fam^2(S')$ is a function $f : S \to S'$ such that the following
    diagram commutes:

    \begin{center}
        \begin{tikzcd}
            S \arrow[rr, "f"] \arrow[d, "\phi"] && S' \arrow[d, "\psi"] \\
            \Fam^2(S) \arrow[rr, "\Fam^2(f)"] && \Fam^2(S')
        \end{tikzcd}
    \end{center}

    Note that the functorial structure on $\Fam$ acts by postcomposition on the second component: a map $X \to Y$ is
    lifted to a map $\sum_I (I \to X) \to \sum_I (I \to Y)$.  The condition thus states that for any $s : S$, the
    commands of $\phi$ at $s$ are equal to the commands of $\psi$ at $f(s)$, the responses are equal for every command,
    and the next states commute with $f$.  This, unfortunately, is too restrictive to work as a set of morphisms for our
    purposes.

    \subsection{Interaction Structure Morphisms}

    Given interaction structures $\mathcal{I} = (S, C, R, t)$ and $\mathcal{J} = (S', C', R', t')$ a morphism $\alpha :
    \mathcal{I} \to \mathcal{J}$ represents the ability to simulate the interactions provided by $\mathcal{I}$ by
    interactions provided by $\mathcal{J}$.  In other words, every command $c : C$ must have a corresponding command
    $\alpha(c) : C'$, and every response of $r : R'_{\alpha(c)}$ must have a corresponding response $\alpha(r) : R_c$.

    The interesting question is how the state types $S$ and $S'$ must relate to each other.  We present two possible
    approaches: when we require a function $S \to S'$ and when we require a relation between $S$ and $S'$.

    \subsubsection{Functional Morphisms}

    \begin{definition}
        A \emph{functional morphism} $\mathcal{I} \to \mathcal{J}$ is a tuple $(\alpha_S, \alpha_C, \alpha_R)$, where
        $\alpha_S : S \to S'$, for every $s : S$, $\alpha_C : C_s \to C'_{\alpha_S(s)}$, and for every $c : C$, $\alpha_R
        : R'_{\alpha_C(c)} \to R_c$, such that the following diagram commutes for every $s : S$ and every $c : C_s$:
        \begin{center}
            \begin{tikzcd}
                R_{c} \arrow[d, "t"] && R'_{\alpha_C(c)} \arrow[ll, "\alpha_R"] \arrow[d, "t'"] \\
                S \arrow[rr, "\alpha_S"] && S'
            \end{tikzcd}
        \end{center}
    \end{definition}

    We will often use $\alpha$ to refer to each of the components of this tuple.

    It is easy to see that every interaction structure has a corresponding identity morphism, and that the composition
    of two functional morphisms is again a functional morphism.

    \subsubsection{Relational Morphisms}

    \begin{definition}
        A \emph{relational morphism} $\mathcal{I} \to \mathcal{J}$ is a tuple $(\alpha_S, \alpha_{C, s, s'}, \alpha_R)$
        such that
        \begin{enumerate}
            \item $\alpha_S$ is a relation between $S$ and $S'$;
            \item for every $s : S$ and $s' : S'$ such that  $\alpha_S(s, s')$, $\alpha_{C, s, s'} : C_s \to C'_{s'}$;
            \item for every $s : S$ and $s' : S'$ such that $\alpha_S(s, s')$, and for every $c : C_s$, $\alpha_R :
                R'_{\alpha_{C, s, s'}(c)} \to R_c$;
            \item and\ldots responses get mapped to related states by $t$ and $t'$.  
        \end{enumerate}
    \end{definition}
    \section{Indexed Monads}

    \section{Free Monads of Interaction Structures}

    \section{Semantic Translations}

\end{document}
